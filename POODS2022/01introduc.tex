% Prof. Dr. Ausberto S. Castro Vera
% UENF - CCT - LCMAT - Curso de Ci\^{e}ncia da Computa\c{c}\~{a}o
% Campos, RJ,  2022
% Disciplina: Paradigma de Desenvolvimento Orientado a Objetos
% Aluno:


\chapterimage{sistemas.png} % Table of contents heading image
\chapter{ Introdu\c{c}\~{a}o}

O \textit{Sistema de comunicação Distribuído} é um sistema que tem como objetivo descrever a troca de informações entre dois pontos. O processo de transmissão e recepção de informações é chamado de comunicação. Os principais elementos de comunicação são o Transmissor de informação, Canal ou meio de comunicação e o Receptor de informação de modo a descentralizar as conversas virtuais, de maneira que os chats podem ser acessados por todos os 27 estados do Brasil conectados através de torres de comunicação.
A empresa propõe que seus serviços sejam disponibilizados para todas as cidades do país 24 horas por dia, conectando pessoas que necessitam se comunicar pela internet. Portanto, sendo Sistema Distribuído uma coleção de sistemas de computador autônomos que são fisicamente separados, mas conectados por uma rede de computadores centralizada que é equipada com software de sistema distribuído. Os computadores autônomos se comunicarão entre cada sistema compartilhando recursos e arquivos e executando as tarefas atribuídas a eles. Após a conclusão bem-sucedida dessas solicitações, o sistema irá conectar os usuários que solicitaram a conexão \cite{devasishakula503_2022}.
 As refer\^{e}ncias bibliogr\'{a}ficas b\'{a}sicas deste projeto s\~{a}o: \cite{Dennis2014}, \cite{Engholm2013}, \cite{Guedes2011},  \cite{Sommerville2018} e \cite{Wazlawick2011}. Como bibliografia complementar ser\~{a}o considerados: \cite{Satzinger2012}, \cite{Shelly2012} e  \cite{Furgeri2013}.
Esses mesmos serviços da empresa só são desligados quando há necessidade de manutenção, medidas de precaução e suprimentos. Este sistema é gerenciado remotamente por desenvolvedores e colaboradores sem a necessidade de colaboradores em um local central. Esses funcionários se comunicam por meio de um sistema interno projetado para manter a comunicação segura e rápida entre os funcionários.

Além disso, o desenvolvimento deste projeto também visa conectar os serviços da empresa. Esses serviços serão desenvolvidos por uma empresa terceirizada que já os projetam para receber o software de inteligência artificial desenvolvido desde o início pela empresa desenvolvedora do sistema de suporte inteligente. A conexão desses serviços funcionará a partir de torres de comunicação que estão espalhadas por todo o Brasil e essas torres são conectadas a satélites espalhados por todo o planeta.

Dessa forma, o acesso a satélites de comunicação e recursos em nuvem é contratado por empresas terceirizadas.
É importante ressaltar que os recursos de projeto, desenvolvimento e manutenção desses serviços são provenientes de empresas terceirizadas.

Este projeto tem como foco a implantação desta tecnologia no Brasil. Aqui, entre outras coisas, são apresentados seus requisitos, recursos, casos de uso, componentes, orçamentos para viabilizar a realização deste sistema.



Neste primeiro capítulo de abertura descreve, e elabora o projeto a ser desenvolvido.



\section{Escopo ou Contextualiza\c{c}\~{a}o do Sistema OO}
Esta seção apresenta informações sobre o sistema, prioridades e justificativas para atingir seus objetivos gerais de facilitar a comunicação entre os serviços da empresa, seus usuários, e facilitar as transações. A partir de um sistema que agiliza o processo de comunicação dos serviços da empresa e encaminha esses recursos para o usuário final.
\subsection{Sistema}
Um dos principais objetivos do projeto é oferecer aos usuários uma agilidade no processo de encaminhamento de suas mensagens, esse servicos será desenvolvido seguindo princípios da \textit{Programação Orientada a Objetos}, onde possam ter uma boa experiência com um sistema mais inteligente.
Dessa forma é possível, por meio do sistema, solicitar o conexões de usuários mais próximo que atenda às necessidades do usuário, bem como planejar conversas, consultar conversas anteriores, dar feedback sobre o serviço prestado pela empresa e se cadastrar no sistema.


\subsection{Sistema de gerenciamento de conversas}
Sistema de comunicação distribuído dos serviços e gestão dos recursos da empresa; Os chats. Desta forma, propõe-se o desenvolvimento de um sistema que permita esta integração, de forma a facilitar e agilizar os processos diários. Este sistema também visa proporcionar à empresa um ambiente de comunicação remoto pensado para manter a comunicação segura e rápida entre os usuários. Além disso, podem ser consultadas informações sobre as conversas dos usuários, como tempo de uso, manutenção, conversar já realizadas, localizações, status, conversa atual, contatos, pendências.


\section{Objetivo do Sistema}
O objetivo geral do projeto, é oferecer recursos para o usuário, na qual possa proporcionar
uma boa experiência com o sistema. Será possível através do sistema, realizar o cadastro
e visualizar mensagens.

\begin{itemize}


     \item \textbf{Objetivos especificos:};
           \subitem - Adicionar amigos em conversar e será possível ser visualizado posteriormente.
           \subitem - Permitir realizar entrar em contato com os usuários presentes no sistema.
           \subitem - Permitir realizar o login e cadastro de usuários.
\end{itemize}


\section{Justificativa}

O sistema visa buscar uma descentralização das conversas virtuais para que os chats possam ser acessados por todos os 27 estados brasileiros. Dessa forma, é possível, por meio do sistema, solicitar as conexões de usuários mais próximas que atendam às necessidades do usuário, bem como planejar conversas, consultar discussões anteriores, dar feedback sobre o serviço prestado pela empresa e se cadastrar no sistema. Este sistema é gerenciado remotamente por desenvolvedores e colaboradores sem a necessidade de colaboradores em um local central. Esses funcionários se comunicam por meio de um sistema interno projetado para manter a comunicação segura e rápida entre os funcionários, a partir de um sistema que agiliza o processo de comunicação dos serviços da empresa e encaminha esses recursos para o usuário final.
Desse modo, devido às necessidades de comunicação presentes no país. Será desenvolvido o sistema proposto anteriormente.
