% Prof. Dr. Ausberto S. Castro Vera
% UENF - CCT - LCMAT - Curso de Ci\^{e}ncia da Computa\c{c}\~{a}o
% Campos, RJ,  2022 
% Disciplina: Paradigma de Desenvolvimento Orientado a Objetos
% Aluno:

\chapterimage{requisitos.jpg} % Table of contents heading image
\chapter{Requisitos do Sistema OO}


Neste cap\'{\i}tulo \'{e} apresentado listas, defini\c{c}\~{o}es e especifica\c{c}\~{o}es de Requisitos do sistema ser desenvolvido. Os requisitos s\~{a}o declara\c{c}\~{o}es abstratas de alto n\'{\i}vel sobre os \textit{servi\c{c}os} que o sistema deve prestar \`{a} organiza\c{c}\~{a}o, e as \textit{restri\c{c}\~{o}es} sobre as quais deve operar. Os requisitos sempre refletem as necessidades dos clientes do sistema.

Sobre os requisitos,  Raul S. Wazlawick afirma:
\begin{citadireta}
A \textit{etapa de levantamento de requisitos} corresponde a buscar todas as informa\c{c}\~{o}es poss\'{\i}veis sobre as fun\c{c}\~{o}es que o sistema deve executar e as restri\c{c}\~{o}es sobre as quais o sistema deve operar. O produto dessa etapa ser\'{a} o documento de requisitos, principal componente do anteprojeto de software.

A \textit{etapa de an\'{a}lise de requisitos} serve para estruturar e detalhar os requisitos de forma que eles possam ser abordados na fase de elabora\c{c}\~{a}o para o desenvolvimento  de outros elementos como casos de uso, classes e interfaces.

O levantamento de requisitos \'{e} o processo de descobrir quais s\~{a}o as \textit{fun\c{c}\~{o}es} que o sistema deve realizar e quais s\~{a}o as \textit{restri\c{c}\~{o}es} que existem sobre estas fun\c{c}\~{o}es  \cite{Wazlawick2011}.
\end{citadireta}



     \section{ Requisitos Funcionais}

             \begin{itemize}
               \item Subsistema AAA
                     \begin{enumerate}
                       \item
                       \item
                       \item
                       \item
                       \item
                     \end{enumerate}
               \item Subsistema BBB
                     \begin{enumerate}
                       \item
                       \item
                       \item
                       \item
                       \item
                     \end{enumerate}
               \item Subsistema CCC
                     \begin{enumerate}
                       \item
                       \item
                       \item
                       \item
                       \item
                     \end{enumerate}
               \item Subsistema DDD
                     \begin{enumerate}
                       \item
                       \item
                       \item
                       \item
                       \item
                     \end{enumerate}
             \end{itemize}


     \section{ Requisitos N\~{a}o-Funcionais}

           \subsection{Requisitos de Usabilidade }
           %%%%%%%%%%%------------------------
           \begin{enumerate}
             \item
             \item
             \item
             \item
             \item
           \end{enumerate}


           \subsection{Requisitos de Confiabilidade }
           %%%%%%%%%%%------------------------
           \begin{enumerate}
             \item
             \item
             \item
             \item
             \item
           \end{enumerate}


           \subsection{Requisitos de Disponibilidade }
           %%%%%%%%%%%------------------------
           \begin{enumerate}
             \item
             \item
             \item
             \item
             \item
           \end{enumerate}


           \subsection{Requisitos de Privacidade }
           %%%%%%%%%%%------------------------
            \begin{enumerate}
             \item
             \item
             \item
             \item
             \item
           \end{enumerate}


           \subsection{Requisitos de Acesso }
           %%%%%%%%%%%------------------------
            \begin{enumerate}
             \item
             \item
             \item
             \item
             \item
           \end{enumerate}


           \subsection{Requisitos de}
           %%%%%%%%%%%------------------------
            \begin{enumerate}
             \item
             \item
             \item
             \item
             \item
           \end{enumerate}


           \subsection{Requisitos de  }
           %%%%%%%%%%%------------------------
            \begin{enumerate}
             \item
             \item
             \item
             \item
             \item
           \end{enumerate}



     \section{Requisitos de Neg\'{o}cios}
     Requisitos do neg\'{o}cio s\~{a}o requisitos de alto n\'{\i}vel que explicam e justificar qualquer projeto. Os requisitos de neg\'{o}cios s\~{a}o as atividades cr\'{\i}ticas de uma empresa que devem ser executadas para atender ao(s) objetivo(s) organizacional(is) enquanto permanecem independentes do sistema solu\c{c}\~{a}o.

            \begin{enumerate}
             \item Reduzir as vendas processadas erroneamente em 30\% at\'{e} o final do ano.
             \item Incrementar o n\'{u}mero de atendimentos online a clientes em 5\% cada m\^{e}s.
             \item
             \item
             \item
           \end{enumerate}





