% Prof. Dr. Ausberto S. Castro Vera
% UENF - CCT - LCMAT - Curso de Ci\^{e}ncia da Computa\c{c}\~{a}o
% Campos, RJ,  2022
% Disciplina: Paradigma de Desenvolvimento Orientado a Objetos
% Aluno:


\chapterimage{sistemas.png} % Table of contents heading image
\chapter{ Introdu\c{c}\~{a}o}

A \textit{Kalender} é um sistema que tem como objetivo possibilitar o agendamento de tarefas de seu usuário. O processo de cadastro de atividades possibilita que o usuário mantenha um histórico das suas atividades diárias. Um calendário é um gráfico ou dispositivo que exibe a data e o dia da semana e, muitas vezes, todo um ano específico dividido em meses, semanas e dias. Dessa maneira, possibilitamos os usuários do site a manter suas atividades sem perder os horários ali cadastrados.
Nossa missão é ser um calendário online que permite que um ou mais usuários editem e, opcionalmente, compartilhem com outros usuários o acesso online a um calendário.

A partir dos nossos serviços poderá fazer uso de uma ferramenta indispensável. Pois, nossas vidas diárias são gerenciáveis apenas graças aos nossos calendários em nossos computadores e telefones. Um calendário online permite-nos gerir os nossos horários e ajuda-nos na gestão do tempo. Pense em todas as vezes que as pessoas se esquecem de suas reuniões e compromissos agendados. As consequências incluem remarcar as reuniões para uma data muito posterior ou até mesmo pagar uma taxa por faltar a um compromisso. Um calendário online permite que você sempre acompanhe suas reuniões e compromissos.

A empresa propõe que seus serviços sejam disponibilizados para todas as cidades do país 24 horas por dia, conectando pessoas que necessitam se comunicar pela internet. Portanto, sendo Sistema Distribuído uma coleção de sistemas de computador autônomos que são fisicamente separados, mas conectados por uma rede de computadores centralizada que é equipada com software de sistema distribuído. Os computadores autônomos se comunicarão entre cada sistema compartilhando recursos e arquivos e executando as tarefas atribuídas a eles. Após a conclusão bem-sucedida dessas solicitações, o sistema irá conectar os usuários que solicitaram a conexão \cite{devasishakula503_2022}.
As refer\^{e}ncias bibliogr\'{a}ficas b\'{a}sicas deste projeto s\~{a}o: \cite{Dennis2014}, \cite{Engholm2013}, \cite{Guedes2011},  \cite{Sommerville2018} e \cite{Wazlawick2011}. Como bibliografia complementar ser\~{a}o considerados: \cite{Satzinger2012}, \cite{Shelly2012} e  \cite{Furgeri2013}.
Esses mesmos serviços da empresa só são desligados quando há necessidade de manutenção, medidas de precaução e suprimentos. Este sistema é gerenciado remotamente por desenvolvedores e colaboradores sem a necessidade de colaboradores em um local central. Esses funcionários se comunicam por meio de um sistema interno projetado para manter a comunicação segura e rápida entre os funcionários.

Além disso, o desenvolvimento deste projeto também visa conectar os serviços da empresa. Esses serviços serão desenvolvidos por uma empresa terceirizada que já os projetam para receber o software de inteligência artificial desenvolvido desde o início pela empresa desenvolvedora do sistema de suporte inteligente. A conexão desses serviços funcionará a partir de torres de comunicação que estão espalhadas por todo o Brasil e essas torres são conectadas a satélites espalhados por todo o planeta.

Dessa forma, o acesso a satélites de comunicação e recursos em nuvem é contratado por empresas terceirizadas.
É importante ressaltar que os recursos de projeto, desenvolvimento e manutenção desses serviços são provenientes de empresas terceirizadas.

Este projeto tem como foco a implantação desta tecnologia no Brasil. Aqui, entre outras coisas, são apresentados seus requisitos, recursos, casos de uso, componentes, orçamentos para viabilizar a realização deste sistema.



Neste primeiro capítulo de abertura descreve, e elabora o projeto a ser desenvolvido.


\section{Escopo ou Contextualiza\c{c}\~{a}o do Sistema OO}
Esta seção apresenta informações sobre o sistema, prioridades e justificativas para atingir seus objetivos gerais de facilitar a organização de tarefas entre os serviços da empresa, seus usuários, e facilitar as transações. A partir de um sistema que agiliza o processo de comunicação dos serviços da empresa e encaminha esses recursos para o usuário final.
\subsection{Sistema}
Um dos principais objetivos do projeto é oferecer aos usuários uma agilidade no processo de organização das suas atividades, esse servicos será desenvolvido seguindo princípios da \textit{Programação Orientada a Objetos}, onde possam ter uma boa experiência com um sistema mais inteligente.
Dessa forma é possível, por meio do sistema, solicitar o conexões de usuários mais próximo que atenda às necessidades do usuário, bem como planejar atividades, consultar atividades anteriores, dar feedback sobre o serviço prestado pela empresa e se cadastrar no sistema.


\subsection{Sistema de gerenciamento de conversas}
Sistema de agenda dos serviços e gestão dos recursos da empresa; As agendas. Desta forma, propõe-se o desenvolvimento de um sistema que permita esta integração, de forma a facilitar e agilizar os processos diários. Este sistema também visa proporcionar à empresa um ambiente de conexão remoto pensado para manter a agilidades segura e rápida entre os usuários. Além disso, podem ser consultadas informações sobre as atividades dos usuários, como tempo de uso, manutenção, tarefas já realizadas, localizações, status, tarefas atuais, contatos, pendências.


\section{Objetivo do Sistema}
O objetivo geral do projeto, é oferecer recursos para o usuário, na qual possa proporcionar
uma boa experiência com o sistema. Será possível através do sistema, realizar o cadastro
e visualizar atividades.

\begin{itemize}


      \item \textbf{Objetivos especificos:};
            \subitem - Adicionar atividades em agenda e será possível ser visualizado posteriormente.
            \subitem - Permitir realizar cadastro de novas atividades no sistema, e visualizar as de outros usuários presentes no sistema.
            \subitem - Permitir realizar o login e cadastro de usuários.
\end{itemize}


\section{Justificativa}

O sistema visa buscar uma descentralização das atividades de maneira virtual para que os calendários possam ser acessados por todos os 27 estados brasileiros. Dessa forma, é possível, por meio do sistema, solicitar as conexões de usuários mais próximas que atendam às necessidades do usuário, bem como planejar atividades, consultar atividades anteriores, dar feedback sobre as tarefas prestadas pela empresa e se cadastrar no sistema. Este sistema é gerenciado remotamente por desenvolvedores e colaboradores sem a necessidade de colaboradores em um local central. Esses funcionários se conectam por meio de um sistema interno projetado para manter a comunicação segura e rápida entre os funcionários, a partir de um sistema que agiliza o processo de atividades dos serviços da empresa e encaminha esses recursos para o usuário final.
Desse modo, devido às necessidades de gerenciamento  de atividades presentes no país. Será desenvolvido o sistema proposto anteriormente.
