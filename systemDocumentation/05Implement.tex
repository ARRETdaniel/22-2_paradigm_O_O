
% Prof. Dr. Ausberto S. Castro Vera
% UENF - CCT - LCMAT - Curso de Ci\^{e}ncia da Computa\c{c}\~{a}o
% Campos, RJ,  2022
% Disciplina: Paradigma de Desenvolvimento Orientado a Objetos
% Aluno:

\chapterimage{Implement.jpg} % Table of contents heading image
\chapter{Implementa\c{c}\~{a}o do Sistema OO}

Este capítulo apresentará a implementação do sistema como as classes que foram desenvolvidas, os módulos que foram utilizados e a documentação do sistema com instruções para o usuário instalar e utilizar o sistema.

\section{Programa\c{c}\~{a}o}
Para a construção do sistema foram utilizadas 3 tecnologias principais:
\begin{itemize}
  \item Framework Ruby on Rails: criação da API (Interface de Programação de Aplicações) que disponibiliza o acesso aos dados.
  \item Ruby: Ruby é uma linguagem de programação interpretada, de alto nível e de propósito geral que suporta vários paradigmas de programação.
  \item SQLite: Biblioteca desenvolvida em C, que funciona como um banco de dados relacional.
\end{itemize}


\subsection{Classes}

Na modelagem do sistema desenvolvido foram definidas mais classes, no entanto, devido à complexidade e a fim de atender aos requisitos do trabalho proposto, somente foram implementadas as apresentadas a seguir:


\begin{lstlisting}[language=Ruby, caption=Ruby Controller]
class ConsultationsController < ApplicationController
  before_action :set_consultation, only: %i[ show edit update destroy ]
### new line below
  before_action :authenticate_user!, except: [:show, :index]
  # GET /consultations or /consultations.json
  def index
    @consultations = Consultation.all
  end

  # GET /consultations/1 or /consultations/1.json
  def show
  end

  # GET /consultations/new
  def new
    @consultation = Consultation.new
  end

  # GET /consultations/1/edit
  def edit
  end

  # POST /consultations or /consultations.json
  def create
    @consultation = Consultation.new(consultation_params)
     ## new line
   # @consultation.user = current_user

    respond_to do |format|
      if @consultation.save
        format.html { redirect_to consultation_url(@consultation), notice: "Consultation was successfully created." }
        format.json { render :show, status: :created, location: @consultation }
      else
        format.html { render :new, status: :unprocessable_entity }
        format.json { render json: @consultation.errors, status: :unprocessable_entity }
      end
    end
  end

  # PATCH/PUT /consultations/1 or /consultations/1.json
  def update
    respond_to do |format|
      if @consultation.update(consultation_params)
        format.html { redirect_to consultation_url(@consultation), notice: "Consultation was successfully updated." }
        format.json { render :show, status: :ok, location: @consultation }
      else
        format.html { render :edit, status: :unprocessable_entity }
        format.json { render json: @consultation.errors, status: :unprocessable_entity }
      end
    end
  end

  # DELETE /consultations/1 or /consultations/1.json
  def destroy
    @consultation.destroy

    respond_to do |format|
      format.html { redirect_to consultations_url, notice: "Consultation was successfully destroyed." }
      format.json { head :no_content }
    end
  end

  private
    # Use callbacks to share common setup or constraints between actions.
    def set_consultation
      @consultation = Consultation.find(params[:id])
    end

    # Only allow a list of trusted parameters through.
    def consultation_params
      params.require(:consultation).permit(:title, :description, :start_time, :end_time)
    end
end



\end{lstlisting}

\begin{lstlisting}[language=Ruby, caption=Ruby LandsController]
  class LandsController < ApplicationController
  def landpage
    @consultations = Consultation.where(
      start_time: Time.now.beginning_of_month.beginning_of_week..Time.now.end_of_month.end_of_week
    )
  end
end

\end{lstlisting}

\subsection{Código Fonte}

Aqui, serão apresentados alguns blocos de código extremamente importantes para o funcionamento do software. Após o login no sistema, a primeira página com a qual o usuário interage é a tela inicial. Esta página é composta por vários componentes e todos eles são "chamados" no código abaixo:


\begin{lstlisting}[language=Ruby, caption=Landpage]

<div class="container mt-5 text-center">
  <h1>Kalender</h1>
  <p>Seja Bem-vindu:</p>

  <h5> <%= current_user.email.split('@')[0].capitalize %> </h5>


<form action="http://[::1]:3000/consultations/new">
<button class="button1 button1">Criar nova atividade!</button>
</form>

  <%= month_calendar(events: @consultations) do |date, consultations| %>
    <div class="day">
      <%= date %>
    </div>
    <% consultations.each do |consultation| %>
      <div class="card-header">
        <h5 class="card-title">
          <%= link_to consultation.title, consultation %>
        </h5>
      </div>
      <div class="card-body">
        <p class="card-text">
          <%= consultation.description %>
        </p>
      </div>
      <div class="card-footer">
        <p class="card-text">
          From: <%= consultation.start_time.strftime('%A, %B %d, %Y at %I:%M %p') %>
        </p>
        <p class="card-text">
          To: <%= consultation.end_time.strftime('%A, %B %d, %Y at %I:%M %p') %>
        </p>
      </div>
    <% end %>
  <% end %>
</div>




\end{lstlisting}

\subsection{Banco da Dados}


\begin{lstlisting}[language=Ruby, caption=CreateConsultations]

class CreateConsultations < ActiveRecord::Migration[7.0]
  def change
    create_table :consultations do |t|
      t.string :title
      t.text :description
      t.datetime :start_time
      t.datetime :end_time

      t.timestamps
    end
  end
end



\end{lstlisting}

\begin{lstlisting}[language=Ruby, caption=AddAdminToUser]
class AddAdminToUser < ActiveRecord::Migration[7.0]
  def change
    add_column :users, :admin, :boolean, default: false
  end
end





\end{lstlisting}

\begin{lstlisting}[language=Ruby, caption=DeviseCreateUsers]

# frozen_string_literal: true

class DeviseCreateUsers < ActiveRecord::Migration[7.0]
  def change
    create_table :users do |t|
      ## Database authenticatable
      t.string :email,              null: false, default: ""
      t.string :encrypted_password, null: false, default: ""

      ## Recoverable
      t.string   :reset_password_token
      t.datetime :reset_password_sent_at

      ## Rememberable
      t.datetime :remember_created_at

      ## Trackable
      # t.integer  :sign_in_count, default: 0, null: false
      # t.datetime :current_sign_in_at
      # t.datetime :last_sign_in_at
      # t.string   :current_sign_in_ip
      # t.string   :last_sign_in_ip

      ## Confirmable
      # t.string   :confirmation_token
      # t.datetime :confirmed_at
      # t.datetime :confirmation_sent_at
      # t.string   :unconfirmed_email # Only if using reconfirmable

      ## Lockable
      # t.integer  :failed_attempts, default: 0, null: false # Only if lock strategy is :failed_attempts
      # t.string   :unlock_token # Only if unlock strategy is :email or :both
      # t.datetime :locked_at


      t.timestamps null: false
    end

    add_index :users, :email,                unique: true
    add_index :users, :reset_password_token, unique: true
    # add_index :users, :confirmation_token,   unique: true
    # add_index :users, :unlock_token,         unique: true
  end
end





\end{lstlisting}

\subsection{Código Fonte - About Us}

\begin{lstlisting}[language=Ruby, caption=About Us]
<div class="about-section">
  <h1>Kalendar</h1>

  <p class="div">Nossa missao e ser um calendario online que permite que um ou mais usuarios editem e, opcionalmente, compartilhem com outros usuarios o acesso online a um calendario.

A partir dos nossos servicos podera fazer uso de uma ferramenta indispensavel. Pois, nossas vidas diarias sao gerenciaveis apenas gracas aos nossos calendarios em nossos computadores e telefones. Um calendario online permite-nos gerir os nossos horarios e ajuda-nos na gestao do tempo. Pense em todas as vezes que as pessoas se esquecem de suas reunioes e compromissos agendados. As consequencias incluem remarcar as reunioes para uma data muito posterior ou ate mesmo pagar uma taxa por faltar a um compromisso. Um calendario online permite que voce sempre acompanhe suas reunioes e compromissos.</p>

</div>

<h2 style="text-align:center">Nosso Time</h2>
<div class="row">
  <div class="column">
    <div class="card">
      <img src="https://media-exp1.licdn.com/dms/image/C4E03AQFXwSHsXB5A-w/profile-displayphoto-shrink_200_200/0/1653855378664?e=1671062400&v=beta&t=JnZi5-8DhyHSwJWl68lXvud1KkoxXQDeGJyuVxGWdRY" alt="Daniel" style="width:5%">
      <div class="container">
        <h2>Daniel Terra Gomes</h2>
        <p class="title">Estudante</p>
        <p>Computer Science | Machine Learning Enthusiast | AI Ethics</p>
        <p>danielterra@pq.uenf.br</p>
        <p><button class="button">Contato</button></p>
      </div>
    </div>
  </div>

  <div class="column">
    <div class="card">
      <img src="https://media-exp1.licdn.com/dms/image/C5603AQHKXoHRT58ayQ/profile-displayphoto-shrink_200_200/0/1516814846280?e=1671062400&v=beta&t=cic_dbtWwgtjWRRnokSRsCxRugh6w52FOZYAksyRxEs" alt="Ausberto" style="width:5%">
      <div class="container">
        <h2>Ausberto S. Castro Vera</h2>
        <p class="title">Professor</p>
        <p>PhD Computer Science: Software Engineering, Requirements Engineering, Computer Security, Cloud Computing</p>
        <p>ascv@uenf.br</p>
        <p><button class="button">Contato</button></p>
      </div>
    </div>
  </div>


</div>

\end{lstlisting}



\section{Documenta\c{c}\~{a}o do Software}

Manual de Instalação
\begin{itemize}
  \item \textbf{Manual de Instala\c{c}\~{a}o}: O qu\'{e} \'{e} necess\'{a}rio para instalar o sistema desenvolvido (banco de dados, vers\~{a}o, bibliotecas, etc.)
  \item \textbf{Manual do usu\'{a}rio}: Os pasos b\'{a}sicos para utilizar o sistema (inicializar, salvar, imprimir, etc.)
  \item \textbf{Outros}....
\end{itemize}
Software Version
\begin{itemize}
  \item Ruby version: ruby-3.1.2
  \item SQLite version:  1.4
\end{itemize}



Configuration
\begin{itemize}
  \item detele Gemfile.lock


  \item update Ruby version of Gemfile to your version


  \item then Bundle update
  \item bundle install

\end{itemize}

Database initialization:  rails db:migrate









Seguindo essas etapas, agora você deve conseguir acessar o sistema no localhost. Quaisquer dúvidas relacionadas ao funcionamento do sistema podem ser esclarecidas na seção Interface do Usuário.
